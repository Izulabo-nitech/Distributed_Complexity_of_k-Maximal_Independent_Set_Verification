%%
%% H27年度情報工学科卒業研究概要スタイルファイル
%%
\documentclass[a4j,twoside]{jarticle}
\bibliographystyle{unsrt}

%%% jdummy.def
%
\DeclareRelationFont{JY1}{mc}{it}{}{OT1}{cmr}{it}{}
\DeclareRelationFont{JT1}{mc}{it}{}{OT1}{cmr}{it}{}
\DeclareFontShape{JY1}{mc}{m}{it}{<5> <6> <7> <8> <9> <10> sgen*min
    <10.95><12><14.4><17.28><20.74><24.88> min10
    <-> min10}{}
\DeclareFontShape{JT1}{mc}{m}{it}{<5> <6> <7> <8> <9> <10> sgen*tmin
    <10.95><12><14.4><17.28><20.74><24.88> tmin10
    <-> tmin10}{}
\DeclareRelationFont{JY1}{mc}{sl}{}{OT1}{cmr}{sl}{}
\DeclareRelationFont{JT1}{mc}{sl}{}{OT1}{cmr}{sl}{}
\DeclareFontShape{JY1}{mc}{m}{sl}{<5> <6> <7> <8> <9> <10> sgen*min
    <10.95><12><14.4><17.28><20.74><24.88> min10
    <-> min10}{}
\DeclareFontShape{JT1}{mc}{m}{sl}{<5> <6> <7> <8> <9> <10> sgen*tmin
    <10.95><12><14.4><17.28><20.74><24.88> tmin10
    <-> tmin10}{}
\DeclareRelationFont{JY1}{mc}{sc}{}{OT1}{cmr}{sc}{}
\DeclareRelationFont{JT1}{mc}{sc}{}{OT1}{cmr}{sc}{}
\DeclareFontShape{JY1}{mc}{m}{sc}{<5> <6> <7> <8> <9> <10> sgen*min
    <10.95><12><14.4><17.28><20.74><24.88> min10
    <-> min10}{}
\DeclareFontShape{JT1}{mc}{m}{sc}{<5> <6> <7> <8> <9> <10> sgen*tmin
    <10.95><12><14.4><17.28><20.74><24.88> tmin10
    <-> tmin10}{}
\DeclareRelationFont{JY1}{gt}{it}{}{OT1}{cmbx}{it}{}
\DeclareRelationFont{JT1}{gt}{it}{}{OT1}{cmbx}{it}{}
\DeclareFontShape{JY1}{mc}{bx}{it}{<5> <6> <7> <8> <9> <10> sgen*goth
    <10.95><12><14.4><17.28><20.74><24.88> goth10
    <-> goth10}{}
\DeclareFontShape{JT1}{mc}{bx}{it}{<5> <6> <7> <8> <9> <10> sgen*tgoth
    <10.95><12><14.4><17.28><20.74><24.88> tgoth10
    <-> tgoth10}{}
\DeclareRelationFont{JY1}{gt}{sl}{}{OT1}{cmbx}{sl}{}
\DeclareRelationFont{JT1}{gt}{sl}{}{OT1}{cmbx}{sl}{}
\DeclareFontShape{JY1}{mc}{bx}{sl}{<5> <6> <7> <8> <9> <10> sgen*goth
    <10.95><12><14.4><17.28><20.74><24.88> goth10
    <-> goth10}{}
\DeclareFontShape{JT1}{mc}{bx}{sl}{<5> <6> <7> <8> <9> <10> sgen*tgoth
    <10.95><12><14.4><17.28><20.74><24.88> tgoth10
    <-> tgoth10}{}
\DeclareRelationFont{JY1}{gt}{sc}{}{OT1}{cmbx}{sc}{}
\DeclareRelationFont{JT1}{gt}{sc}{}{OT1}{cmbx}{sc}{}
\DeclareFontShape{JY1}{mc}{bx}{sc}{<5> <6> <7> <8> <9> <10> sgen*goth
    <10.95><12><14.4><17.28><20.74><24.88> goth10
    <-> goth10}{}
\DeclareFontShape{JT1}{mc}{bx}{sc}{<5> <6> <7> <8> <9> <10> sgen*tgoth
    <10.95><12><14.4><17.28><20.74><24.88> tgoth10
    <-> tgoth10}{}
\DeclareRelationFont{JY1}{gt}{it}{}{OT1}{cmr}{it}{}
\DeclareRelationFont{JT1}{gt}{it}{}{OT1}{cmr}{it}{}
\DeclareFontShape{JY1}{gt}{m}{it}{<5> <6> <7> <8> <9> <10> sgen*goth
    <10.95><12><14.4><17.28><20.74><24.88> goth10
    <-> goth10}{}
\DeclareFontShape{JT1}{gt}{m}{it}{<5> <6> <7> <8> <9> <10> sgen*tgoth
    <10.95><12><14.4><17.28><20.74><24.88> tgoth10
    <-> tgoth10}{}
\endinput
%%%% end of jdummy.def

 % フォント関連のエラー対策(らしい)

%\usepackage{graphicx}
\usepackage[dvipdfmx]{graphicx}
\usepackage{amsmath,amsthm}		% math系
\usepackage{amssymb}			% math系
%\usepackage{float}				% 図表の挿入箇所を固定する[H]指定
%\usepackage{cite}				% 参考文献
%\usepackage{url}				% 参考文献中のURL表記
\usepackage{algorithm}			% アルゴリズム環境
\usepackage{algorithmic}		% アルゴリズム環境
\usepackage{comment}			% コメントアウト環境
\usepackage{bm}	%太字形式のベクトル

% 定理環境
\newtheorem{theorem}{定理}
\newtheorem{lemma}{補題}
\newtheorem{definition}{定義}


% Unix上でのコンパイルはthesis_abst-jisを利用してください.
\usepackage{thesis_abst}

% マージンはプリンタによって変更
\addtolength{\oddsidemargin}{0mm}
\addtolength{\evensidemargin}{0mm}

% baselinestretchを変更すると上部枠の大きさが変わるのでおすすめしない
\renewcommand{\baselinestretch}{1}

\種別{修了}  % この行を消してはいけない
\学籍番号{31414050}
\氏名{佐藤 僚祐}
%\英語氏名{} %未使用
\研究室{金・片山}
\系{ネットワーク} % 学生が所属する系を記入.教員(研究室)の所属する系ではない
\題目{密でないグラフに対して最大k-plexを発見する \\ 準指数時間アルゴリズム} % 途中で改行 "\\" を挿入可
\年度{2} % !=年 発表は2月です
\begin{document}              % この行を消してはいけない
\twocolumn[\vspace*{9mm}]     % この行を消してはいけない
\begin{論文概要}              % この行を消してはいけない
\setcounter{page}{2}          % 表(左綴じ)は1、裏(右綴じ)は2を指定
%%%%%%% ここからアブスト本体 %%%%%%%

\section{はじめに}
近年,ソーシャルネットワークにおいて大きな凝集性
のある部分グラフを見つけることは大きな研究の
トピックとなっている.その応用はアドホックネットワーク \cite{chen2004clustering} 
やデータマイニング \cite{washio2003state} など多岐の分野にわたっている.
凝集性のある部分グラフのうち最も一般的に使用
されているモデルはクリークである.クリークとは全ての
2頂点間に辺がある部分グラフで,最も凝集性のある
グラフとみることができる.グラフ中の最も大きいサイズの
クリークを見つける最大クリーク問題はグラフアルゴリ
ズムにおける基本的な問題である.しかし,クリークは
制約が厳しく扱いづらい面がある.そこで,さまざまな
クリークの緩和モデルが提案された.$k$-plexは
クリークの緩和モデルの一つであり,辺の数に関して
緩和されており$k=1$のとき1-plexはクリークと同
義である.グラフ中の最も大きいサイズの$k$-plexを
見つける問題を最大$k$-plex問題という.最大
$k$-plex問題はNP完全であり,頂点数$n$に対して
$n$の多項式時間で解くことができないことが知られ
ている.そこで,最大$k$-plex問題を高速に解くための
さまざまなアルゴリズムが提案されている.提案された
アルゴリズムの多くは$2^{n}n^{O(1)}$時間で動く.

今回,我々はグラフの辺の本数$m$に対して最大$k$-plex
問題を$O(n^{k}2^{\sqrt{m}})$時間で解く
アルゴリズムを提案する.このアルゴリズムは$m$の
準指数時間で動くため,辺の本数が少ない,密でない
グラフに対する最大k-plexの発見に有用である.  

\section{諸定義}
グラフの構造を次のように定義する.グラフ$G=(V,E)$は
頂点数$n=|V|$と辺の本数$m=|E|$を持つ単純無向
グラフとする.グラフ$G$の頂点部分集合$S$によって
誘導される誘導部分グラフを$G[S]$と表す.頂点$v$に
辺が接続されているとき,$v$に隣接しているといい,
$v$に隣接している頂点を$v$の近傍と呼び,$N(v)$と
表す.頂点$(u,v)$間の最短パスの本数を頂点$(u,v)$の
距離とする.

\section{最大$k$-plex問題の \\ 準指数時間アルゴリズム}
我々が提案する最大$k$-plexの発見アルゴリズムは,
最大クリーク問題を$n2^{O(\sqrt{m})}$で解くことが
できるという既存の結果を応用している. \cite{fomin2010exact}
以下に本研究のアルゴリズムとその実行時間を示す.

最小次数の頂点$v$を選ぶ($v<\sqrt{m}$が仮定できる).
$G$中の最大サイズの$k$-plexである$S$を見つける
ために2つの部分問題にブランチする.
\begin{enumerate}
 \item $v$が$S$中の頂点の1つである
 \item $v$が$S$中の頂点の1つでない
\end{enumerate}

1つ目の部分問題では$S$に含まれる可能性のある
頂点部分集合を探索して$v$を含んだ
最大$k$-plexを見つける.補題により距離2以上離れている
頂点は$k$つ以上は$S$に含まれないことが示せるので,
$S$を選ぶのに必要な頂点数は
\[ 2^{M} \times  \sum_{i = 1}^{k - 1}\binom{N}{i}  \]
である($M$:$v$の近傍の頂点数,$N$:$v$から距離2以上
離れた頂点数).$M < \sqrt{m}$,$N \leqq n$ であり,また 
$ \sum_{i = 1}^{k}\binom{n}{i} \leqq ({en}/{k})^{k}$
であることが知られているので必要なステップ数は多くとも
$({en}/{k})^{k}2^{\sqrt{m}}$ステップである.

2つ目の部分問題では$v$を$G$から削除してアルゴリズムを
再帰的に呼び出す.再帰の回数は多くとも$n-1$回である.

ゆえにアルゴリズムのステップ数は
\[  {\left( \frac{en}{k} \right)}^{k}2^{\sqrt{m}} +  (n - 1){\left( \frac{en}{k} \right)}^{k}2^{\sqrt{m}} =n{\left( \frac{en}{k} \right)}^{k}2^{\sqrt{m}} \]
でありその実行時間は 
\[ O(n{\left( \frac{en}{k} \right)}^{k}2^{\sqrt{m}}) = O(n^{k}2^{\sqrt{m}}) \]
である.

\section{まとめと今後の課題}
今回の研究によって,最大$k$-plex問題を$O(n^{k}2^{\sqrt{m}})$時間で
解くアルゴリズムを得ることができた.このアルゴリズムでは
$S$に入る可能性のない頂点も探索しているので,これを改良して
$O(n^{k}2^{\sqrt{m}})$の境界を突破できるかが今後の課題となる.

\bibliography{b4sato}

%%%%%% 以下の行は消さないこと %%%%%%%

\clearpage                       % この行を消すと最終ページの枠線消滅の危機
\end{論文概要}                   % この行を消してはいけない
\end{document}                   % この行を消してはいけない
