\documentclass[12pt]{thesis}
\bibliographystyle{unsrt}
\input{jdummy.def} 				% フォント関連のエラー対策(らしい)
\usepackage[dvipdfmx]{graphicx}
\usepackage{amsmath}			% math系
\usepackage{amssymb}			% math系
%\usepackage{float}				% 図表の挿入箇所を固定する[H]指定
\usepackage{cite}				% 参考文献
%\usepackage{url}				% 参考文献中のURL表記
\usepackage{algorithm}			% アルゴリズム環境
\usepackage{algorithmic}			% アルゴリズム環境
\usepackage{comment}			% コメントアウト環境
\usepackage{bm}					%太字形式のベクトル
\usepackage{amsthm}			%定理用?

%%% 泉先生がコメントをつける用 %%%
\usepackage[normalem]{ulem}
\usepackage{color}
\newcommand{\Izumi}[1]{\textcolor{blue}{(#1)}}
\newcommand{\Izurep}[2]{\textcolor{red}{\sout{#1}}{\Izumi{#2}}}

\headsep=1.4cm  %本文上にスペースを空けたい場合は 20mm にする

\newcommand{\CONGEST}{\textsf{CONGEST}}

% 定理環境
\usepackage{amsthm} %定理用
\theoremstyle{definition}
\newtheorem{theorem}{定理}[chapter]
\newtheorem{lemma}{補題}[chapter]
\newtheorem{definition}{定義}[chapter]
\newtheorem{fact}{事実}[chapter]
\newtheorem*{prf*}{証明}
%\renewcommand{\theproof}{}
%\newcommand{\qed}{\hfill$\square$\par}

%%%%%%%% ここから本体 %%%%%%%%%%%%%%%%%%%%%%%%

\begin{document}
\baselineskip=22pt
\pagestyle{empty}

% タイトル
%\gradyear{2020}
\papertitleJP{$k$-極大独立集合検証問題の \\ 分散計算複雑性}
\papertitleEN{Distributed Complexity of $k$-Maximal \\ Independent Set Verification}
\studentID{31414050}
\degree{master}
\program{cs}
\labo{片山・金}
\enteryear{2019}
\name{佐藤 僚祐}
\maketitle

% 目次
\pagestyle{myheadings}	% ページ番号を右上につける
\pagenumbering{roman}	% ページ番号をローマ数字で
\tableofcontents

\newpage

% 本文
\pagenumbering{arabic}	% ページ番号をアラビア数字で

\chapter{はじめに}

\section{研究背景}
\Izumi{句点と読点は全角の,と.にする}
分散グラフアルゴリズムとは,計算機を頂点,辺を通信リンクとみなしてネットワークをモデル化したグラフ上において,
そのネットワーク自身を入力としてグラフに対して定義される諸問題を解く枠組みである.分散アルゴリズムにおける
代表的なモデルのひとつとして{{\CONGEST}}モデルが存在する.{\CONGEST}モデルにおいて,
各ノードは同期ラウンドに従って実行され,メッセージ交換によって協調動作を行う.
各ノードは各ラウンドで(i)$b$ビットのメッセージを隣接ノードに送信
(ii)隣接ノードからメッセージを受信(iii)内部計算の3つの動作をする.
特に{\CONGEST}モデルでは$b = O(\log n)$が仮定される.
{\CONGEST}モデルにおいて,ある1つのノードにグラフ全体のトポロジの情報を集め,そのノード上で
逐次アルゴリズムを実行するという素朴なアプローチから,任意の問題に対して自明に
$O (n^{2})$ラウンドの上界を得ることができる.{\CONGEST}モデルにおける下界の証明では,
下界が$\Omega(n^{2})$ラウンドにどれだけ近づけることができるかに興味がある.

本研究ではグラフ上の最適化問題の一つである,最大独立点集合に注目する.{\CONGEST}モデルにおいて,
ネットワーク上の最大独立集合あるいはその近似解を発見する数多くの分散グラフアルゴリズムが研究されている.
%各頂点が隣接していない頂点部分集合を独立集合といい,最大独立集合とは頂点数が最も多い独立集合である.
大きなサイズの独立集合は経済学,計算生物学,符号理論,実験計画法など様々な分野への応用に用いられる
 \cite{kawarabayashi2019improved} が,そもそも最大独立集合問題はNP完全であり,頂点数$n$に対して
$n$の多項式時間で解くことは,その近似を含めて絶望的であるとされている.分散グラフアルゴリズムの文脈からは,
各ノードのローカル計算において指数時間の計算を許した場合における計算ラウンド数の複雑性に関して近年
議論が進んでおり,最大独立集合の発見あるいはその近似解の発見に対して上界,下界の両面から,いくつかの
結果が知られている.
(\Izumi{ここ,具体的にどのような上下界が得られているか説明を入れる}).
例えば,{\CONGEST}モデルにおいて最大重み付き独立集合の$(1 + \epsilon) \cdot \Delta$-近似
($\Delta$は頂点の最大次数) を高確率で$\left(poly(\log \log n)/\epsilon \right)$ラウンドで
発見するアルゴリズム\cite{kawarabayashi2019improved} や,最大独立集合を発見するアルゴリズムに対する
$\Omega \left(\frac{n^{2}}{(\log n)^{2}}\right)$ラウンドの下界 \cite{censor2017quadratic},
最大独立集合の$(\frac{3}{4} + \epsilon)$-近似を発見するアルゴリズムに対する
$\Omega \left(\frac{n^{2}}{(\log n)^{3}}\right)$ラウンドの下界 \cite{efron2020beyond} などが知られている.
逐次アルゴリズムにおける計算困難性から,指数時間を要する内部計算を許容して分散グラフ
アルゴリズムの複雑性の議論を行うことの妥当性にはやや疑問が残る.

\section{本研究の目的と結果}
そこで今回,我々は最大独立集合の局所最適解である$k$-極大独立集合
($k$-Maximal Independet Set, $k$-MIS)について考える.
\Izumi{$k$-MISはそれほど一般的な概念ではないので,ここにラフな説明ほしい}
独立集合のうち,$k$個の頂点をその集合から取り除いて独立集合を維持したまま$k + 1$個以上の頂点を
追加することができないとき,その独立集合を$k$-MISという.
与えられたグラフ$G=(V, E)$に対する$k$-MISの発見は,自然な局所探索により実現することが可能である.すなわち,
$I$を現在構成されている独立点集合とすると,任意のサイズ$k$以下の部分集合$I' \subseteq I$ およびサイズ$k + 1$の
部分集合$U \subseteq V\setminus I$について,$(I \setminus I') \cup U$が独立点集合となるかどうかを確認し,
そのような$I'$,$U$が存在すれば$I$を更新することを繰り返すことで,最終的に$k$-MISを得ることができる.
$k = O(1)$の場合この処理は$n$の多項式時間内で収まることが容易にわかる.一方で,この更新処理,すなわち,
与えられた独立点集合$I$が既に$k$-MISを構成しているかどうか検証する問題が{\CONGEST}モデル上で効率的に
解けるかどうかは明らかではない.そこで,本研究では$k$-MISの検証問題(verification)に着目し,その複雑性について
議論を行う.具体的には,本研究は以下に挙げる結果を示す\Izumi{下に結果をitemizeの箇条書きでまとめる}.
\begin{itemize}
\item {\CONGEST}モデルにおいて,1-MIS検証問題は$O(1)$ラウンドで解くことができる.
\item {\CONGEST}モデルにおいて,2-MIS検証問題には$\tilde{\Omega} (\sqrt{n})$ラウンドの下界が存在する.
\item {\CONGEST}モデルにおいて,3-MIS検証問題には$\tilde{\Omega} (n)$ラウンドの下界が存在する.
\item {\CONGEST}モデルにおいて,任意の$l \geq 1$に対して$k = 4l + 5$としたときに$k$-MIS検証問題には
$\Omega\left(n^{2 - \frac{1}{l+1}}/l\right)$ラウンドの下界が存在する.
\end{itemize}

\section{関連研究} 
{\CONGEST}モデルにおける最大独立集合問題の通信複雑性としては,最大重み付き独立集合の
$(1 + \epsilon) \cdot \Delta$-近似($\Delta$は頂点の最大次数) を高確率で見つける
アルゴリズムに対する$\left(poly(\log \log n)/\epsilon \right)$ラウンドの上界
\cite{kawarabayashi2019improved} や,最大独立集合を見つけるアルゴリズムに対する
$\Omega \left(\frac{n^{2}}{(\log n)^{2}}\right)$ラウンドの下界 \cite{censor2017quadratic},
最大独立集合の$(\frac{1}{2} + \epsilon)$-近似を見つけるアルゴリズムに対する
$\Omega \left(\frac{n}{(\log n)^{3}}\right)$ラウンドの下界,
$(\frac{3}{4} + \epsilon)$-近似を見つけるアルゴリズムに対する
$\Omega \left(\frac{n^{2}}{(\log n)^{3}}\right)$ラウンドの下界 \cite{efron2020beyond} が知られている.

また集中型アルゴリズムについて,$P = NP$が成り立たない限り,任意の$\epsilon > 0$に対して
最大独立集合の$n^{\frac{1}{2} - \epsilon}$近似を発見するアルゴリズムは存在しないことが知られている \cite{haastad1999clique}.
 
本研究の下界に関する結果は2者間通信の枠組みにおける交叉判定問題からの帰着に基づいているが,
交叉判定問題からの帰着によって下界を示すという証明方法は多くの問題に対して用いられている.
一部の例として最小カット発見問題と最小全域木問題に対する$\Omega (D + \sqrt{n})$の下界
($D$はグラフの直径) \cite{sarma2012distributed}や部分グラフ$H_{k}$検出問題に対する
$\Omega \left(\frac{n^{2 - 1/k}}{bk}\right)$の下界 \cite{fischer2018possibilities},
近似最大クリーク$K_{l}$検出問題に対する$\Omega \left(\frac{n}{(l + \sqrt{n})b}\right)$の下界
 \cite{czumaj2020detecting}などが挙げられる.

\Izumi{集中型アルゴリズムの近似不可能性の結果も,具体的な近似率限界を挙げて説明する.また,CONGEST上の
極大独立点集合の結果(すなわち0-MISの結果)も挙げること.s}

\section{本研究の成果}
今回,我々は$k$-MIS検証問題に対するいくつかの複雑性を示した.その結果を表\ref{tab: k-MIS}にまとめた.
最初に,1-MIS検証問題が$O(1)$ラウンドで解けることを証明した.
次に,2-MIS検証問題に対する$\tilde{\Omega} (\sqrt{n})$ラウンドの下界と
3-MIS検証問題に対する$\tilde{\Omega} (n)$ラウンドの下界を証明した.
最後に,任意の$l \geq 1$に対して$k = 4l + 5$としたときに$k$-MIS検証問題に対する
$\Omega\left(n^{2 - \frac{1}{l+1}}/l\right)$ラウンドの下界を証明した.
特に,下界の証明のアイデアは2者間通信の枠組みにおける交叉判定問題からの帰着に基づいている.

\begin{table}[htb]
  \begin{center}
    \caption{本研究が示した$k$-MIS検証問題のラウンド複雑性}
    \begin{tabular}{|c||c|c|c|c|} \hline
      $k$ & 1 & 2 & 3 & $4l + 5$ \\ \hline
      上界or下界 & 上界 & 下界 & 下界 & 下界 \\ \hline
      結果 & $O(1)$ & $\tilde{\Omega} (\sqrt{n})$ & $\tilde{\Omega} (n)$ & $\Omega\left(n^{2 - \frac{1}{l+1}}/l\right)$ \\ \hline
    \end{tabular}
    \label{tab: k-MIS}
  \end{center}
\end{table}


\section{論文の構成}
本論文は全5章で構成される.第2章ではグラフの構造と用語の定義をしている.
第3章では1-MIS検証問題に対する$O(1)$ラウンドアルゴリズムについて述べている.
第4章では$k$-MIS検証問題($k = 2, 3, 4l + 5 ( l \geq 1)$)に対する下界について述べている.
第5章ではまとめについて述べている.

\chapter{諸定義}

\section{$CONGEST$モデル}
本研究で考える$CONGEST$モデルは,単純無向グラフ連結グラフ$G = (V, E)$により表現される.ここで$V$はノードの集合で
$|V| = n$とし,$E$は通信リンクの集合である.$CONGEST$モデルでは計算機はラウンドに従って同期して動作するものとする.
1ラウンド内で,隣接頂点へのメッセージ送信,隣接頂点からのメッセージ受信,内部計算を行う.各辺は単位ラウンドあたり
$b = O(\log n)$ビットを双方向に伝送可能であり,各ノードは同一ラウンドに異なる接続辺に異なるメッセージを
送信可能である.また,各ノードには$O(\log n)$ビットの自然数値によるIDが付与されており,
自身の隣接ノードすべてのIDを既知であるとする.各ノードはグラフのトポロジに関する事前知識を持たないものとする.

\section{2者間通信複雑性}
2者間通信複雑性の枠組みでは,アリスとボブの二人のプレイヤーがそれぞれ$k$ビットの0/1のデータ列で構成される
プライベートな入力$x$および$y$を持っているとする.プレイヤーの目標は,結合関数$f(x, y)$を計算することであり,
複雑性の尺度として$f(x, y)$を計算するためにアリスとボブが通信によって交換する必要のあるビット数が用いられる.

この枠組みにおける重要な問題として,交叉判定問題(set-disjointness)がある.この問題では,アリスとボブは
それぞれ$x \in \{0, 1\}^{k}$と$y \in \{0, 1\}^{k}$を入力として持ち,目的は
$DISJ_{k} (x, y) :=\bigvee_{i = 1}^{k} x_{i} \land y_{i}$を計算することである.$k$ビットの交叉判定問題を解くために,
アリスとボブは通信によって$\Omega (k)$ビット交換する必要があることが知られており 
\cite{kalyanasundaram1992probabilistic},この事実を用いて最小カット発見問題や最小全域木問題 \cite{sarma2012distributed},
部分グラフ検出問題や \cite{fischer2018possibilities} ,近似最大クリーク検出問題 \cite{czumaj2020detecting} といった
さまざまな問題に対する下界の証明がされている.

{\CONGEST}モデルにおいて,入力グラフ上に特性$P$があるかどうかの判定に対する下限の証明を
2者間交叉判定問題から帰着するアプローチは以下のとおりである.
最初にアリスとボブは特殊なグラフ$G = (V, E)$の構築と$G$を$G_{A}$と$G_{B}$に分割するカット辺$C$の決定を行う.
次に,アリスとボブは入力文字列に基づいてそれぞれ$G_{A}$に辺$E_{A}$と頂点$V_{A}$,
$G_{B}$に辺$E_{B}$と頂点$V_{B}$を追加する.
このとき,$DISJ_{k} (x, y)=1$のときのみ,何らかの特性$P$(例えば,$P$:「グラフに与えられたMISが2-MISでない」)を持つように
辺や頂点を追加する.また,カット辺$C$は入力文字列に依存しないようにする.グラフ$G$に辺や頂点を追加したグラフを
$G^{x, y} = (V', E')$とすると
$V' = V \cap (V_{A} \cap V_{B}), E' = E \cap (E_{A} \cap E_{B})$表すことができる.グラフ$G^{x, y}$の構造の概要を図 \ref{Gxy} に示す.

\begin{figure}[ht]
\begin{center}
\includegraphics[width=120mm]{Gxy.png}
\end{center}
\caption{$G^{x, y} = (V', E')$}
\label{Gxy}
\end{figure}

アリスとボブは,入力グラフ上に特性$P$があるかを判定する分散アルゴリズムをシミュレートできる.
アリスは$G_{A}$に含まれる頂点を,ボブは$G_{B}$に含まれる頂点をシミュレートする.
2者間通信複雑性モデルでのシミュレートは,次のように実行される.$G_{A}$中の辺で送信されるメッセージ,あるいは
$G_{B}$中の辺で送信されるメッセージは,アリスとボブがそれぞれお互いと通信せずにシミュレートできる.カット辺$C$を通じて
送信されるメッセージに対しては,お互い情報を交換する必要がある.{\CONGEST}モデルにおいてグラフ上に特性$P$が
あるかどうかを$r$ラウンドで判定するアルゴリズム$\mathcal{A}$が存在したとすると,アリスとボブは特性$P$の判定のために
$O(r \cdot |C| \cdot \log n)$ビット通信したことになる.これは,各ラウンドで,アルゴリズムが各辺で$O(\log n)$ビットの通信を
行っているからである.このグラフにおいてアルゴリズム$\mathcal{A}$を実行すると同時に2者間交叉判定問題も解けていることになる.
例えばアルゴリズムを実行した結果,入力グラフに特性$P$があると判定されれば$DISJ_{k} (x, y)=1$であるとわかるからである.
交叉判定問題の通信複雑性よりアリスとボブは少なくとも$\Omega (k)$ビットは通信しているはずである.
したがって,$CONGEST$モデルにおいてと特性$P$があるかどうかを判定する任意のアルゴリズムに対して
$r = \Omega (k / |C| \cdot \log n)$ラウンドの下界を得ることができる.カット辺の大きさが小さくなるほど下界が強くなる.
%そのままにしました

\section{$k$-極大独立集合}
\begin{comment}
\begin{definition}
$I$が$k$-極大独立集合 $ \Leftrightarrow$ \\
\begin{math}
\lnot \left( \exists I' \subseteq I |I'| = k,  \\ \exists S \subseteq V |S|\geq k + 1, (I\setminus I')\cap Sが独立集合 \right)
\end{math}
\end{definition}
\end{comment}

\begin{definition}
頂点集合$I$に対して,以下を満たす頂点集合$I' \subseteq I$と$S\subset V \backslash I$のペアが
存在しないとき,Iを$k$-極大独立集合と呼ぶ.
\begin{enumerate}
\item $|I'| = k$
\item $|S| \geq k + 1$
\item $(I \backslash I') \cup S$は独立集合
\end{enumerate}
\end{definition}
つまり,ある独立集合$I$に対して,サイズ$k$の$I$の部分集合$I'$を取り除いてサイズ$k + 1$以上の$V$の部分集合$S$を
$I$に追加したものが新たな独立集合になり得ないとき,$I$を$k$-極大独立集合と定義する.


\newpage

\chapter{1-MIS検証問題の$O(1)$ラウンドアルゴリズム}
この章では,1-MIS検証問題を$O(1)$ラウンドで解くアルゴリズムついて述べる. \\
{\CONGEST}モデルにおいて,入力としてグラフ$G$と独立集合$I$が与えられたとき,
1-MIS検証問題を解くために次のようなアルゴリズム$\mathcal{A}$を実行する.
\begin{enumerate}
\item 各頂点$v \in I$は,自分のIDである$v$.idを隣接頂点全員に送信する.
\item 各頂点$u \notin I$のうち,2種類以上のIDをもらった頂点はアルゴリズムから離脱する.
\item 離脱しなかった頂点$u \notin I$のうち1種類だけのID($v$.idとする)を受信した頂点は
離脱していない全隣接頂点へ$v$.idを送信する.
\item 各頂点$v \in I$は,自身と同じ$v$.idを返信してきた頂点の集合($v.X$とする)のサイズを数え,
そのサイズ$|v.X|$を$v.X$中の頂点に送信する.自身と違う$v$.idを返信してきた頂点に対しては何もしない.
\item メッセージを受け取った$v.X$中の頂点は,送られたサイズ$|v.X| - 1$と3番目のステップで$v$.idを
送信した頂点の個数を比較する.全ての$v \in I$に対する$v.X$中の全頂点でその値が等しければ
与えられた独立集合$I$は1-MISであり,一つでも等しくなければ$I$は1-MISでないと判定する.
\end{enumerate}
%メッセージを受け取ったv.X中の頂点は, v.Xがクリークを成しているか確認する.
%確認は,送られたサイズ|v.X| - 1 と,3番目のステップでv.idを送信した頂点の個数を比較して,
%v.X中の全頂点でその値が等しければクリークを成していると判断する.等しくなければ,
%クリークを成していないとする.あるvについて後者が成り立つときは,v.Xの中にサイズ2の独立点集合があり,
%vとそれを入れ替えることが可能なので,1-misでない.そうでないとき
%(=すべてのvについてv.Xがクリークを成しているとき),1-misである
アルゴリズム$\mathcal{A}$について,以下の定理が成り立つ.
\begin{theorem}
$CONGEST$モデルにおいて,アルゴリズム$\mathcal{A}$は1-MIS検証問題を$O(1)$ラウンドで解く.
\end{theorem}
\begin{prf*}
とちゅう
\end{prf*}
\newpage

\chapter{$k$-MIS検証問題の下界}
この章では,$k$-MIS検証問題の下界についての議論を行う.
4.1節では,2-MIS検証問題の下界についての定理とその証明を述べる.
4.2節では,3-MIS検証問題の下界についての定理とその証明を述べる.
4.3節では,$k$-MIS検証問題($k = 4l + 5, l \geq 1$)の下界についての定理とその証明を述べる.

\section{2-MIS検証問題の下界}

この節では2-MIS検証問題の下界についての議論を行う.具体的には,次の定理を証明する.
\begin{theorem}
{\CONGEST}モデルにおいて,2-MIS検証問題を解く全てのアルゴリズムは$\tilde{\Omega} (\sqrt{n})$の
通信ラウンド数を必要とする.
\end{theorem}
\begin{prf*}
まず初めにアリスとボブが構築するグラフ$G = (V, E)$を図 \ref{2_G}に示す. 

\begin{figure}[ht]
\begin{center}
\includegraphics[width=120mm]{2_G.png}
\end{center}
\caption{$G = (V, E)$}
\label{2_G}
\end{figure}

図中の頂点のうち灰色のものは独立集合に含まれる頂点とする.図中の辺を定式化すると以下のようになる.
%図中の頂点のうち,灰色のものは独立集合に含まれる頂点とする.図中で省略されている辺と$K_{W(y)}$の構造は次のとおりである.
\begin{itemize}
\item $\forall i((a_{i}^{1}, b_{i}^{1}), (a_{i}^{2}, b_{i}^{2})) \in E$
\end{itemize}

このグラフが「$G$中に与えられている独立集合が,$DISJ_{N \times N} (x, y) = 1$のときのみ2-MISでない」という特性(*)を持つように,
$G_{A}$に構造$H_{A}$,$G_{B}$に構造$H_{B}$を追加する.
(なお,$DISJ_{N \times N} (x, y) :=\bigvee_{i = 1}^{N} \bigvee_{j = 1}^{N} x_{i, j} \land y_{i, j}$で定義される.)
$H_{A}$と$H_{B}$の中身は以下のとおりである.

\newpage
\begin{itemize}
\item $H_{A}$: $(a_{i}^{1}, a_{j}^{2}) \in E_{A} \Leftrightarrow x_{i, j} = 0$
\item $H_{B}$: $W(y)$頂点のクリーク$K_{W(y)}$($W(y)$は0/1のデータ列$y$中に含まれる1の個数を表す.)
$K_{W(y)}$中の頂点$c_{i, j} \in V_{B}$は$y_{i, j} = 1$であるような$(i, j)$でインデックスづけされるものとする. \\
このとき,$(c_{i, j}, b_{i}^{1}) \in E_{B}$ かつ $(c_{i, j}, b_{j}^{2}) \in E_{B}$
\end{itemize}

$G = (V, E)$に辺と頂点を追加したグラフ$G^{x, y} = (V', E')$を図 \ref{2_G(x,y)}に示す.

\begin{figure}[ht]
\begin{center}
\includegraphics[width=120mm]{2_Gxy.png}
\end{center}
\caption{$G^{x, y} = (V', E')$}
\label{2_G(x,y)}
\end{figure}

このグラフ$G^{x, y} = (V', E')$が上記の特性(*)を満たしていることを示すために,次の2点を確認する. \\
(i)$DISJ_{N \times N} (x, y) = 1$のとき,グラフに与えられている独立集合が2-MISでない: \\
$x_{i, j} = y_{i, j} =1$とすると,$b_{i}^{1}$と$b_{j}^{2}$の2点を取り除いて$a_{i}^{1}$, $a_{j}^{2}$, $c_{i, j}$の
3点を追加できることから確認できる. \\
(ii)$DISJ_{N \times N} (x, y) = 0$のとき,グラフに与えられている独立集合が2-MISである: \\ 
グラフに与えられている独立集合が2-MISでないと仮定する.このとき,ある2点を取り除くことで独立集合に追加できる3点が存在する.
2点の取り除き方は(1)$b_{i}^{1}$と$b_{j}^{1}(i \neq j)$, (2)$b_{i}^{2}$と$b_{j}^{2}(i \neq j)$, (3)$b_{i}^{1}$と$b_{j}^{2}$が考えられる.
(1)では$a_{i}^{1}$と$a_{j}^{1}$の2点しか追加できる可能性がなく,(2)では$a_{i}^{2}$と$a_{j}^{2}$の2点しか追加できる可能性がない.
(3)において,$b_{i}^{1}$を取り除いて$a_{i}^{1}$を追加し,$b_{j}^{2}$を取り除いて$a_{j}^{2}$を追加し,さらに$c_{i, j}$を追加することを考える.
$a_{i}^{1}$と$a_{j}^{2}$が両方とも追加できるのは$x_{i, j} = 1$のときのみであり,$c_{i, j}$が追加できる($c_{i, j}$が存在する)のは
$y_{i, j} = 1$のときのみであるが,これは$DISJ_{N \times N} (x, y) = 0$に矛盾する.したがってグラフに与えられている独立集合から2点取り除いて
3点追加することはできないため,この独立集合は2-MISである.

今回,$N \times N$ビットの交叉判定インスタンスをグラフに埋め込んでおり,カット辺のサイズ$|C| = 2N$であることがわかる.
{\CONGEST}モデルにおいてグラフ上に与えられた独立集合が2-MISであるかどうかを$r$ラウンドで判定する
アルゴリズム$\mathcal{A}$が存在したとすると,アリスとボブは$O(r \cdot |C| \cdot \log n)$ビット通信したことになる.
このグラフにおいてアルゴリズム$\mathcal{A}$を実行すると同時に2者間交叉判定問題も解けていることになるので,
交叉判定問題の通信複雑性よりアリスとボブは少なくとも$\Omega (N \times N)$ビットは通信しているはずである.
よって,$r = \Omega (N / 2\log n) = \tilde{\Omega}(N)$ラウンドの下界を得ることができる.図\ref{2_G(x,y)}からわかる通り,
$A^{1}, A^{2}, B^{1}, B^{2}$はそれぞれ$N$頂点で構成されており,$K_{W(y)}$の頂点数は$O(N^{2})$であるため,
グラフ全体の頂点数$n$は$n = O(N + N^{2})$である.したがって$N = O(\sqrt{n})$になるため,
$\tilde{\Omega}(\sqrt{n})$ラウンドの下界を得ることができる.
\end{prf*}
\newpage

\section{3-MIS検証問題の下界}
この節では3-MIS検証問題の下界についての議論を行う.具体的には,次の定理を証明する.
\begin{theorem}
{\CONGEST}モデルにおいて,3-MIS検証問題を解く全てのアルゴリズムは$\tilde{\Omega} (n)$の通信ラウンド数を必要とする.
\end{theorem}

\begin{prf*}
まず初めにアリスとボブが構築するグラフ$G = (V, E)$を図 \ref{3_G}に示す.

\begin{figure}[ht]
\begin{center}
\includegraphics[width=120mm]{3_G.png}
\end{center}
\caption{$G = (V, E)$}
\label{3_G}
\end{figure}

図中の頂点のうち灰色のものは独立集合に含まれる頂点とする.また,四角で囲まれている部分はクリークを表す.
図中の辺を定式化すると以下のようになる.
\begin{itemize}
\item $A^{1}, A^{2}, B^{1}, B^{2}$は$N$頂点のクリーク$K_{N}$
\item $\forall i((a_{i}^{1}, c_{i}^{1}), (a_{i}^{2}, c_{i}^{2}), (b_{i}^{1}, c_{i}^{1}), (b_{i}^{2}, c_{i}^{2})) \in E$
\item $\forall i((a_{i}^{1}, s), (a_{i}^{2}, s), (b_{i}^{1}, s), (b_{i}^{2}, s)) \in E$
\end{itemize}

このグラフが「$G$中に与えられている独立集合が,$DISJ_{N \times N} (x, y) = 1$のときのみ3-MISでない」という特性(*)を持つように,
$G_{A}$に構造$H_{A}$,$G_{B}$に構造$H_{B}$を追加する.$H_{A}$と$H_{B}$の中身は以下のとおりである.
\newpage
\begin{itemize}
\item $H_{A}$: $(a_{i}^{1}, a_{j}^{2}) \in E_{A} \Leftrightarrow x_{i, j} = 0$
\item $H_{B}$: $(b_{i}^{1}, b_{j}^{2}) \in E_{B} \Leftrightarrow y_{i, j} = 0$
\end{itemize}

$G = (V, E)$に辺を追加したグラフ$G^{x, y} = (V, E')$を図 \ref{3_G(x,y)}に示す.

\begin{figure}[ht]
\begin{center}
\includegraphics[width=120mm]{3_Gxy.png}
\end{center}
\caption{$G^{x, y} = (V, E')$}
\label{3_G(x,y)}
\end{figure}

このグラフ$G^{x, y} = (V, E')$が上記の特性(*)を満たしていることを示すために,次の2点を確認する. \\
(i)$DISJ_{N \times N} (x, y) = 1$のとき,グラフに与えられている独立集合が3-MISでない: \\
$x_{i, j} = y_{i, j} =1$とすると,$s$と$c_{i}^{1}$と$c_{j}^{2}$の3点を取り除いて$a_{i}^{1}$, $b_{i}^{1}$, $a_{j}^{2}$, $c_{j}^{2}$の
4点を追加できることから確認できる。 \\
(ii)$DISJ_{N \times N} (x, y) = 0$のとき,グラフに与えられている独立集合が3-MISである: \\ 
グラフに与えられている独立集合が3-MISでないと仮定する.このとき,ある3点を取り除くことで独立集合に追加できる4点が存在するが,
$A^{1}, A^{2}, B^{1}, B^{2}$がそれぞれクリークであるため,4点を追加するためにはそれぞれから1点を選ぶ必要がある.
$c_{i}^{1}$を取り除いて$a_{i}^{1}$と$b_{i}^{1}$を追加し,$c_{j}^{2}$を取り除いて,$a_{j}^{2}$と$b_{j}^{2}$を独立集合に追加したとする.
$a_{i}^{1}$と$a_{j}^{2}$が両方とも追加できるのは$x_{i, j} = 1$のときのみであり,$b_{i}^{1}$と$b_{j}^{2}$が両方とも追加できるのは
$y_{i, j} = 1$のときのみであるが,これは$DISJ_{N \times N} (x, y) = 0$に矛盾する。したがってグラフに与えられている独立集合から3点取り除いて
4点追加することはできないため,この独立集合は3-MISである.

今回,$N \times N$ビットの交叉判定インスタンスをグラフに埋め込んでおり,カット辺のサイズ$|C| = 4N$であることがわかる.
{\CONGEST}モデルにおいてグラフ上に与えられた独立集合が3-MISであるかどうかを$r$ラウンドで判定する
アルゴリズム$\mathcal{A}$が存在したとすると,アリスとボブは$O(r \cdot |C| \cdot \log n)$ビット通信したことになる.
このグラフにおいてアルゴリズム$\mathcal{A}$を実行すると同時に2者間交叉判定問題も解けていることになるので,
交叉判定問題の通信複雑性よりアリスとボブは少なくとも$\Omega (N \times N)$ビットは通信しているはずである,
よって,$r = \Omega (N / 4\log n) = \tilde{\Omega}(N)$ラウンドの下界を得ることができる.図\ref{3_G(x,y)}からわかる通り,
$A^{1}, A^{2}, B^{1}, B^{2}, C^{1}, C^{2}$はそれぞれ$N$頂点で構成されているため,グラフ全体の頂点数$n$は$n = O(N)$である.
したがって$N = O(n)$になるため,$\tilde{\Omega}(n)$ラウンドの下界を得ることができる..
\end{prf*}
\newpage

\section{$k$-MIS検証問題の下界}
このセクションではk-MIS検証問題の下界についての議論を行う.具体的には,次の定理を証明する.
\begin{theorem}
{\CONGEST}モデルにおいて,任意の$l \geq 1$に対して$k = 4l + 5$としたとき$k$-MIS検証問題を解く全てのアルゴリズムは
$\Omega\left(n^{2 - \frac{1}{l+1}}/l\right)$の通信ラウンド数を必要とする.
\end{theorem}

\begin{prf*}
最初に$l = 1$の場合を考える.

まず初めにアリスとボブが構築するグラフ$G = (V, E)$を図 \ref{k_G}に示す.

\begin{figure}[ht]
\begin{center}
\includegraphics[width=120mm]{k_G.png}
\end{center}
\caption{$G = (V, E)$}
\label{k_G}
\end{figure}

図中の頂点のうち灰色のものは独立集合に含まれる頂点とする.また,四角で囲まれている部分はクリークを表す.
図中の辺を定式化すると以下のようになる.
\begin{itemize}
\item $A^{1}, A^{2}, B^{1, 1}, B^{1, 2}, B^{2, 1}, B^{2, 2}$は$N$頂点のクリーク$K_{N}$
\item $D^{1, 1}, D^{1, 2}, D^{2, 1}, D^{2, 2}$は$\sqrt{N}$頂点のクリーク$K_{\sqrt{N}}$
\item $\forall i((a_{i}^{1}, s), (a_{i}^{2}, s)) \in E$
\item $\forall i, j((a_{k}^{1}, c_{i}^{1, 1}), (a_{k}^{1}, c_{j}^{1, 2}), (a_{k}^{2}, c_{i}^{2, 1}), (a_{k}^{2}, c_{j}^{2, 2}), \in E$
(ただし$k = (i - 1) \times \sqrt{N} + j$)
\item $\forall i, j((b_{k}^{1, 1}, e_{i}^{1, 1}), (b_{k}^{1, 1}, e_{j}^{1, 2}), (b_{k}^{2, 1}, e_{i}^{2, 1}), (b_{k}^{2, 1}, e_{j}^{2, 2}), \in E$
(ただし$k = (i - 1) \times \sqrt{N} + j$)
\item $\forall i, j((b_{k}^{1, 2}, e_{i}^{1, 1}), (b_{k}^{1, 2}, e_{j}^{1, 2}), (b_{k}^{2, 2}, e_{i}^{2, 1}), (b_{k}^{2, 2}, e_{j}^{2, 2}), \in E$
(ただし$k = (i - 1) \times \sqrt{N} + j$)
\item $\forall i((c_{i}^{1, 1}, d_{i}^{1, 1}), (c_{i}^{1, 2}, d_{i}^{1, 2}), (e_{i}^{1, 1}, d_{i}^{1, 1}), (e_{i}^{1, 2}, d_{i}^{1, 2})) \in E$
\item $\forall i((c_{i}^{2, 1}, d_{i}^{2, 1}), (c_{i}^{2, 2}, d_{i}^{2, 2}), (e_{i}^{2, 1}, d_{i}^{2, 1}), (e_{i}^{2, 2}, d_{i}^{2, 2})) \in E$
\end{itemize}

このグラフが「$G$中に与えられている独立集合が,$DISJ_{N \times N} (x, y) = 1$のときのみ
$k( = 4 \times 1 + 5 = 9)$-MISでない」という特性(*)を持つように,
$G_{A}$に構造$H_{A}$,$G_{B}$に構造$H_{B}$を追加する.$H_{A}$と$H_{B}$の中身は以下のとおりである.

\begin{itemize}
\item $H_{A}$: $(a_{i}^{1}, a_{j}^{2}) \in E_{A} \Leftrightarrow x_{i, j} = 0$
\item $H_{B}$: $(b_{i}^{1,1}, b_{j}^{2,1}) \in E_{B} \Leftrightarrow y_{i, j} = 0, (b_{i}^{1,2}, b_{j}^{2,2}) \in E_{B} \Leftrightarrow y_{i, j} = 0$
\end{itemize}

$G = (V, E)$に辺を追加したグラフ$G^{x, y} = (V, E')$を図 \ref{k_Gxy}に示す.

\begin{figure}[ht]
\begin{center}
\includegraphics[width=120mm]{k_Gxy.png}
\end{center}
\caption{$G^{x, y} = (V, E')$}
\label{k_Gxy}
\end{figure}

このグラフ$G^{x, y} = (V, E')$が上記の特性(*)を満たしていることを示すために,次の2点を確認する. \\
(i)$DISJ_{N \times N} (x, y) = 1$のとき,グラフに与えられている独立集合が$k( = 9)$-MISでない: \\
$x_{i, j} = y_{i, j} =1$とすると,$s, c_{k}^{1,1}, c_{l}^{1,2}(i = (k - 1) \times \sqrt{N} + l),
 c_{p}^{2,1}, c_{q}^{2,2}((j = (p - 1) \times \sqrt{N} + q), 
 e_{k}^{1,1}, e_{l}^{1,2}, e_{p}^{2,1}, e_{q}^{2,2}$の9点を取り除いて
 $a_{i}^{1}, a_{j}^{2}, b_{i}^{1,1}, b_{j}^{2,1}, b_{i}^{1,2}, b_{j}^{2,2}, d_{k}^{1,1}, d_{l}^{1,2}, d_{p}^{2,1}, d_{q}^{2,2}$の
10点を追加できることから確認できる. \\
(ii)$DISJ_{N \times N} (x, y) = 0$のとき,グラフに与えられている独立集合が$k( = 9)$-MISである: \\ 
グラフに与えられている独立集合が9-MISでないと仮定する.このとき,ある9点を取り除くことで独立集合に追加できる10点が存在するが, \\
$A^{1}, A^{2}, B^{1,1}, B^{2,1}, B^{1,2}, B^{2,2}, D^{1,1}, D^{1,2}, D^{2,1}, D^{2,2}$がそれぞれクリークであるため,
10点を追加するためにはそれぞれから1点を選ぶ必要がある.
$d_{k}^{1,1}$を追加するために$c_{k}^{1,1}$と$e_{k}^{1,1}$を,$d_{l}^{1,2}$を追加するために$c_{l}^{1,2}$と$e_{l}^{1,2}$を,
$d_{p}^{2,1}$を追加するために$c_{p}^{2,1}$と$e_{p}^{2,1}$を,$d_{q}^{2,2}$を追加するために$c_{q}^{2,2}$と$e_{q}^{2,2}$を独立集合から
取り除くとする.このとき,$i = (k - 1) \times \sqrt{N} + l$を満たす$i$に対して$b_{i}^{1,1}$と$b_{i}^{1,2}$を,
$j = (p - 1) \times \sqrt{N} + q$を満たす$j$に対して$b_{j}^{2,1}$と$b_{j}^{2,2}$を追加するできる可能性がある.さらに,$A^{1}$と$A^{2}$の頂点を
独立集合に追加できるように$s$も取り除くと,同じ$i, j$に対して$a_{i}^{1}$と$a_{j}^{2}$を追加できる可能性をもたせることができる.
しかし,$a_{i}^{1}$と$a_{j}^{2}$が両方とも追加できるのは$x_{i, j} = 1$のときのみであり,$b_{i}^{1,1}$と$b_{j}^{1,2}$(もしくは$b_{i}^{2,1}$と$b_{j}^{2,2}$)が
両方とも追加できるのは$y_{i, j} = 1$のときのみであるが,これは$DISJ_{N \times N} (x, y) = 0$に矛盾する.
したがってグラフに与えられている独立集合から9点取り除いて10点追加することはできないため,この独立集合は9-MISである.

今回,$N \times N$ビットの交叉判定インスタンスをグラフに埋め込んでおり,カット辺のサイズ$|C| = 4 \sqrt{N}$であることがわかる.
{\CONGEST}モデルにおいてグラフ上に与えられた独立集合が$k( = 9)$-MISであるかどうかを$r$ラウンドで判定する
アルゴリズム$\mathcal{A}$が存在したとすると,アリスとボブは$O(r \cdot |C| \cdot \log n)$ビット通信したことになる.
このグラフにおいてアルゴリズム$\mathcal{A}$を実行すると同時に2者間交叉判定問題も解けていることになるので,
交叉判定問題の通信複雑性よりアリスとボブは少なくとも$\Omega (N \times N)$ビットは通信しているはずである.
よって,$r = \Omega \left(N^{3/2} / 4\log n\right) = \tilde{\Omega}\left(N^{3/2}\right)$ラウンドの下界を得ることができる.
図\ref{k_Gxy}からわかる通り,$A^{1}, A^{2}, B^{1,1}, B^{2,1}, B^{1,2}, B^{2,2}$はそれぞれ$N$頂点で,$C, D, E$の頂点集合は
それぞれ$\sqrt{N}$頂点で構成されているため,グラフ全体の頂点数$n$は$n = O(N)$である.
したがって$N = O(n)$になるため,$\tilde{\Omega}\left(n^{3/2}\right)$ラウンドの下界を得ることができる. 

$l \geq 2$については,図\ref{k_Gxy}を図\ref{k_Gxyl}のように拡張する,

\begin{figure}[ht]
\begin{center}
\includegraphics[width=120mm]{k_Gxyl.png}
\end{center}
\caption{$l \geq 2$における$G^{x, y} = (V, E')$}
\label{k_Gxyl}
\end{figure}

$4(l + 1) + 1 = 4l + 5$頂点を取り除くことで$4(l + 1) + 2 = 4l + 6$頂点を追加できるかどうか,
つまりのグラフに与えられている独立集合が$k( = 4l + 5)$-MISであるかどうかの議論を上記と同様に行うことができる.
このときカット辺のサイズは$|C| = 2(l + 1) \cdot N^{1/(l + 1)}$であることがわかるため,
$r = \Omega \left(N^{2 - \frac{1}{l + 1}} / 2( l + 1)\log n\right) = \tilde{\Omega}\left(N^{2 - \frac{1}{l + 1}}/l\right)$ラウンドの
下界を得ることができる.
$A, B$の頂点集合はそれぞれ$N$頂点で,$C, D, E$の頂点集合はそれぞれ$N^{1/(l + 1)}$頂点で構成されているため,
グラフ全体の頂点数$n$は$n = O(N)$である.したがって$N = O(n)$になるため,
$\tilde{\Omega}\left(n^{2 - \frac{1}{l + 1}}/l\right)$ラウンドの下界を得ることができる.
\end{prf*}
\newpage

\chapter{まとめと今後の課題}
\section{まとめ}
本研究では極大独立集合検証問題に対するいくつかの複雑性を示した.
具体的には,1-MIS検証問題に対する$O(1)$ラウンドの上界,
2-MIS検証問題に対する$\tilde{\Omega} (\sqrt{n})$ラウンドの下界,
3-MIS検証問題に対する$\tilde{\Omega} (n)$ラウンドの下界,
$k$-MIS検証問題($k = 4l + 5, l \geq 1$)に対する$\tilde{\Omega}\left(n^{2 - \frac{1}{l + 1}}/l\right)$ラウンドの下界を証明した.

\section{今後の課題}
4.3節で一般の$k$に対する$k$-MIS検証問題の下界を証明したが,$k = 4,...,8$については現在
3-MIS検証問題と同じ下界しか得られていない.この下界をよりタイトにできるかが今後の課題である.
\newpage

\chapter*{謝辞}
本研究の機会を与え,数々の御指導を賜りました泉泰介准教授に深く感謝致します.
また,本研究を進めるにあたり多くの助言を頂き,様々な御協力を頂きました泉研究室
の学生のみなさんに深く感謝致します.

\newpage

\bibliography{M2sato}

\end{document}